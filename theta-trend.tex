\documentclass[12pt,DIV=18]{scrartcl}
\usepackage[utf8]{inputenc}
\usepackage[english, russian]{babel}
\usepackage[unicode=true, hidelinks]{hyperref}
\usepackage[usenames, dvipsnames]{color}
\usepackage{soulutf8}

\title{Theta Trend Trading System}
\author{Dan. thetatrend.com (Перевод: userkilouser@gmail.com)}
\date{September 2014}

\begin{document}

\maketitle
Theta Trend -- Авторская торговая система


\tableofcontents

\section*{Введение и допущения}
\label{intro}
\addcontentsline{toc}{section}{Введение и допущения}
\textcolor{White}{.}

Добро пожаловать и спасибо за покупку \textbf{Theta Trend}. \textbf{Theta Trend} -- это опционная трендовая механическая торговая система (МТС) с положительной тетой (мы продаем опционы). Это значит, что \textbf{Theta Trend} следует за трендом цены [базового актива]. Однако, вместо того чтобы покупать базовый актив, \textbf{Theta Trend} продает опционы вне денег (OTM). Продажа OTM опционов позволяет увеличить количество способов заработать деньги с этой системой. Мы в выигрыше при сильном тренде, слабом тренде, и даже, в некоторых случаях, когда тренд не формируется или ломается. Кроме того, система приносит прибыль с течением времени из-за распада проданных опционов.

\bigskip

Использование \textbf{Theta Trend} предполагает владение базовыми понятиями об опционах и ETF, однако совсем нет необходимости быть эскпертом по волатильности или техническому анализу. Если вы понимаете, как работают опционы и знаете, как построить вертикальный спред, -- вы разберетесь в \textbf{Theta Trend}. \textbf{Theta Trend} предназначена быть практичной, легко реализуемой системой, которая генерирует прибыльные сделки с более высокой вероятностью, чем типичные трендовые системы.

\bigskip

\textbf{Theta Trend} не требует от вас следить за рынком весь день и, определенно, не является \textit{day trading} системой. Большинство сделок длиться от пары недель до месяца и за ними можно наблюдать и управлять в течении нескольких минут в день. Однако важность управления рисками и дисциплину в трейдинге нельзя недооценивать.

\bigskip

Программное обеспечение, которым я лично пользуюсь и очень рекомендую -- это \textbf{Thinkorswim} от \textbf{TD Ameritrade}. Качество их софта -- превосходное и все изображения в этом документе взяты из этой программы.

\bigskip

Успешный трейдинг - это скорее путь, чем цель. Надеюсь, что эта система сохранит вам несколько лет, не тратя их на разработку и разочаровывающие исследования. Это уже сделано мной. Если что-нибудь в системе кажется неясным, -- не стесняйтесь написать на \textcolor{Blue}{\href{mailto:info@thetatrend.com}{info@thetatrend.com}}, \textit{enjoy}.

\bigskip

-Dan\par
\textcolor{Blue}{\href{mailto:info@thetatrend.com}{info@thetatrend.com}}

\section*{Глава 1. Вертикальные сперды}
\label{chapter1}
\addcontentsline{toc}{section}{Глава 1. Вертикальные сперды}
\bigskip

Вертикальный спред является строительным блоком системы \textbf{Theta Trend}. Вертикальный спред называется так из-за природы создания, если глядеть на опционную доску, и может быть дебитовым (вы покупаете спред), или кредитовым (вы продаете спред). Все сделки \textbf{Theta Trend} начинаются с OTM кредитных спредов, так что мы всегда продаем премию. Мы продаем OTM опционы чтобы избежать проблем, связанных с досрочным исполнением [опционов]. Хотя вероятность досрочного исполнения конкретно OTM опционов является довольно малой.

\bigskip

Изображение 1.1 демонстрирует опционную доску для IWM -- ETF для Russel 2000. На доске представлены: \textit{open interest} (открытый интерес), \textit{delta} (дельта), \textit{implied  volatility} (волатильность), \textit{extrinsic} и \textit{theta} (тета). Вертикальный спред строится на опционах с одной датой экспирации и включает в себя продажу близкого к деньгам опциона (ATM) и покупку другого опциона для хеджа риска далеко вне денег. На изображении подсвечены опционы в деньгах (ITM).

\bigskip

Изображение 1.1 Опционы с разными датами экспирации и их цены для IWM (ETF для Russel 2000).

\bigskip

\textbf{Theta Trend} использует медвежий колл-спред для шорта рынка, и бычий пут-спред для лонга. Суть спреда в том, что, если предполагается рост рынка, то мы продаем путы ниже рынка. Если предполагается падение рынка, то мы продаем коллы выше текущей цены.

\bigskip

Изображение 1.2 показывает пример вертикального спреда который может использовать \textbf{Theta Trend} чтобы встать в лонг по рынку. На изображении: красная линия представляет ценовой профиль (risk profile) позиции на день экспирации, а белая линия -- на текущий день. В то время как линия рисков [и доходов] отражает позицию на момент экспирации, управление риском позиции на текущий день является нашей главной заботой. Смысл именно в использовании для управления позицией именно рисков текущего дня. Если вы управляете риском позиции текущего дня, вместо упования на будущее, то это поможет избежать серьезных убытков.

\bigskip

Изображение 1.2 Вертикальный бычий пут-спред в SPY (S\&P 500 ETF) демонстрирует эмуляцию длинной позиции [в базовом активе].

\bigskip

Позиция на Изображении 1.3 показывает пример короткого OTM вертикального колл-спреда, который есть бычья версия трейда, представленного на Изображении 1.2. Этот пример -- реальная сделка. Вы можете видеть, что текущая рыночная цена TLT (ETF на 20+ летние казначейские облигации) -- около \$118 и позиция в одном контракте имеет прибыль примерно \$18. Один из основных принципов системы \textbf{Theta Trend} -- это торговать множество мелких позиций на некореллированных [друг с другом] рынках.

\bigskip

Изображение 1.3 Реальный вертикальный бычий колл-спред в TLT.

\bigskip

Одна из первых особенностей, которую стоит упомянуть о вертикальном спреде -- это его низкий коэффициент риск/прибыль. На изображении, показанном выше, коэффициент риск/прибыль равен 1.6 / 0.4 или 4 к 1. Однако, компромисс в том, что вероятность получить прибыль довольно велика. Мы можем примерно оценить вероятность, что опционы окажутся в деньгах на момент экспирации, глядя на дельту в день продажи [спреда].

\bigskip

\textbf{Theta Trend}, как правило, продает опционы с дельтой 15 -- 40. Опцион с дельтой 15 имеет 15\% шанс оказаться в деньгах на момент экспирации, что означает 100\% - 15\% = 85\% шанс его обесценивания к дате истечения [опциона]. Все в работе с опционами связано компромиссом. Компромисс при продаже опционов далеко вне денег в том, что вы получаете небольшие премии, но с большой вероятностью  обесценивания [опционов к экспирации]. Малые величины [получаемой] премии могут быть из-за неблагоприятного соотношения риск/прибыль или, в некоторых случаях, когда открытая позиция не соответствует критерию риск/прибыль системы \textbf{Theta Trend}. Специфические характеристики отношения риск/прибыль, которые используются системой \textbf{Theta Trend} для открытия позиции обсуждаются в ~\hyperref[chapter5]{\ul{Главе 5} -- Правила открытия позиции}.

\section*{Глава 2. Принципы следования за трендом}
\label{chapter2}
\addcontentsline{toc}{section}{Глава 2. Принципы следования за трендом}
\bigskip

\textbf{Theta Trend} -- это трендовая торговая система. Трендовики обычно зарабатывают, держа прибыльные позиции до окончания тренда, и быстро выходят из убыточных позиций. Большинство трендовиков придерживаются нескольких типов механических торговых систем со строгим управлением рисками и правилами открытия и закрытия позиций. Как говорится, большинство трейдеров пытается следовать за каким-либо типом  тренда. Тренд может образоваться на любом тайм-фрейме -- от секунд до месяцев или лет. Успех большинства трендовых систем базируется на том понятии, что цена актива не адекватна (not normally 
distributed), а должна быть много выше или ниже.

\bigskip

Многие трендовые МТС имеют меньше прибыльных сделок, чем убыточных, но все-таки приносят деньги, т.к. доход от прибыльных сделок много больше убытков. Отношение прибыльных сделок к убыточным как 40\% к 60\% -- это не редкость, и может быть встроено в профитную торговую систему с надлежащим управлением рисками. Система \textbf{Theta Trend} построена так, что добивается увеличенного процента прибыльных сделок. Таким образом, использовать систему психологически легче. \textit{Forward testing} системы \textbf{Theta Trend} показал, что соотношение прибыльных сделок к убыточным как 60\% к 40\% вполне достижимо.

\bigskip

Сделки в системе \textbf{Theta Trend} имеют ограниченный потенциал по прибыли, но и убытки уменьшены и даже возможен доход, если ситсема сто\'{и}т против рынка. \textbf{Theta Trend} призвана быть механической системой, т.е. мнение трейдера не включено в процесс принятия решения [по сделке]. Сделка либо соответствует критериям открытия/закрытия, либо нет. Тренд выявляется индикатором системы, и сделки не открываются против тренда. \textbf{Theta Trend} также предназначена быть в рынке все время, пока вход валиден и нет причин для закрытия.

\bigskip

Не смотря на то, что система \textbf{Theta Trend} не включает в себя мнение трейдера, основанного на [внутри?]дневных данных, все-таки необходимо принятие решение по поводу портфеля инструментов для торговли. Трендовые системы работают на многочисленных некоррелированных рынках т.к. это увеличивает и выравнивает доходы. \textbf{Theta Trend} поддерживает идею, что торговля на некоррелированных рынках увеличивает доходы, и работает на широком спектре рынков: валютном, рынке облигаций, товаров и акций. Примеры портфелей обсуждаются в последующих главах. Главная идея -- это торговать на нескольких некоррелированных друг к другу рынках в надежде, что если вы ошиблись и теряете деньги на одном рынке, а на другом рынке (или ранках) вы правы и это перекрывает убыток.

\bigskip

Несмотря на то, что системой торгуются широкий спектр рынков, вам не нужно разбираться в фундаментальной информации, касающейся всех этих рынков. \textbf{Theta Trend} -- это МТС, предполагающая, что весь фундаментал уже заложен в цене. Сигналы, генерируемые \textbf{Theta Trend}, никогда не берут в расчет анализ или прогнозы по занятости, урожайности или процентной ставке. Кроме того, \textbf{Theta Trend} реагирует на изменения тренда, но не предсказывает тренд. Сделки совершаются в надежде, что цена будет двигаться в нашем направлении. А управление позицией (рисками) базируется на предположении, что цена идет против нас.

\bigskip

Управление рисками -- крайне важная тема и детально обсуждается в~\hyperref[chapter3]{\ul{Главе 3}}. Система \textbf{Theta Trend} так универсальна (works  largely) потому, что риск-менеджмент направлен на избежание больших убытков. Многие опционные системы защищают ``регулирование'' убыточных позиций и предлагают ``пересидеть'' просадку (sitting through the pain). Сделки по системе \textbf{Theta Trend} время от времени также дают просадку, однако в ней ясно определены точки выхода [из позиции]. Система четко определяет завершение тренда, таким образом нет смысла пытаться чинить убыточную позицию. \textbf{Theta Trend} всегда торгует по тренду, и никогда против него.

\bigskip

Цель использования МТС -- минимизировать ошибки трейдера. Эмоционально отстраниться в трейдинге очень тяжело, но это является необходимым условием принятия верных решений. Есть многочисленные ресурсы, обсуждающие психологию в трейдинге и понять личные психологические особенности очень рекомендуется.

\section*{Глава 3. Управление рисками}
\label{chapter3}
\addcontentsline{toc}{section}{Глава 3. Управление рисками}
\bigskip

Обсуждаемые темы:

\begin{itemize}
\item Ожидание
\item Общий риск и риск на сделку
\end{itemize}

\bigskip

\textbf{\ul{Внимание}: Это самая важная часть в торговой системе Theta Trend}. Изучение любой другой части, без строгого соблюдение правил управления рисками, не принесет положительных результатов. Изучение с повторением этой Главы настоятельно рекомендуется.
 
\bigskip
 
\textbf{Theta Trend} и все остальные трендовые системы приносят прибыль в основном за счет управления рисками. \textit{Forward  testing} показывает, что система имеет приблизительно 60\% прибыльных сделок. Однако, в связи с тем, что доходы по позиции ограничены, должен соблюдаться строгий риск-менеджмент для получения положительного ожидания. Положительное ожидание -- это цель всех торговых систем и, по-простому, означает, что система является прибыльной длинном промежутке времени. Для иллюстрации посмотрите ниже на примеры ожидаемых выплат:
 
\bigskip
 
\setlength{\parindent}{0.5cm}
Expected payout = P(winning) -- P(losing)\par
Expected payout = (winning \% * Win) -- (losing \% * Loss)\par
\bigskip
\textbf{Пример 1:} Торговая система с 65\% \textit{Win Rate}, Win = 1, Loss = 2\par
\bigskip
EP = (0.65 * 1) -- (0.35 * 2) = 0.65 -- 0.70 = -0.05\par
\bigskip
\textbf{Пример 2:} Торговая система с 45\% \textit{Win Rate}, Win = 2, Loss = 1\par
\bigskip
EP = (0.45 * 2) -- (0.55 * 1) = 0.90 -- 0.55 = 0.35\par
\bigskip
\textbf{Пример 3:} Торговая система \textbf{Theta Trend} с 60\% \textit{Win Rate}, Win = 1, Loss = 1\par
\bigskip
EP = (0.6 * 1) -- (0.4 * 1) = 0.60 -- 0.40 = 0.20\par

\bigskip

Глядя на эти примеры, мы можем видеть, что высокого \textit{Win Rate} не достаточно для создания прибыльной торговой системы.
 
\bigskip
 
Пример 1 -- идентичен теряющему трейдеру, который берет прибыль в сделках как только она появляется, и надеется, что убыточные сделки выйдут в 0.

\bigskip
 
Пример 2 -- типичная модель следования за трендом. Не смотря на малый процент \textit{Win Rate}, система все еще приносит прибыль с течением времени потому, что профит значительно больше суммы потерь по большому числу мелких убыточных сделок.

\bigskip
 
Пример 3 -- показывает принцип работы \textbf{Theta Trend}. Система \textbf{Theta Trend} сконструирована таким образом, чтобы получать больше прибыльных сделок, чем убыточных и быть в профите, четко управляя потерями. Такая система психологически комфортнее, нежели система, дающая длинные серии убыточных сделок.

\subsection*{Общий риск}

Расчет общего риска -- это отправная точка для риск-менеджмента, и по сути является суммой денег задействованных в торговле. Например, если у вас есть \$10 000 и вы держите резерв 10\%, которым не собираетесь торговать, то ваш общий риск составляет \$9 000. Размер всех позиций расчитывается на основе общего риска, а правила, контролирующие просадку, уменьшают размер позиции, если сумма общего риска снизилась до определенного уровня.

\subsection*{Максимальный риск на сделку}

Допускается максимальный риск на сделку не превышающий 1\% от суммы общего риска, и он зачастую менее 1\%. Общий риск -- это сумма на счету трейдера, которая расчитывается ежемесячно в целях оценки риска позиции. \textbf{Theta Trend} предлагает свой инструмент управления рисками, но вы можете изменить его параметры под свой стиль торговли и допуски по рискам. Более подробно о этом вы можете найти в ~\hyperref[chapter8]{\ul{Главе 8} -- Адаптация системы}.

\subsection*{Максимальный доход и цель по доходу в сделке}

Максимальный доход на сделку равен изначально полученному [при продаже спреда] кредиту. Цель при продаже одного вертикального спреда равна 80\% от максимума. Максимальный убыток по сделке имеет ту же величину. Важно иметь в виду, что ваш потенциальный убыток и потенциальная прибыль должны учитывать комиссии. Комиссионные издержки неотъемлемая часть торговли, и не принятие их в расчет равносильно ведению бизнеса не волнуясь о перерасходах.

\bigskip

Примечание: Есть возможность получить более высокий процент прибыльных сделок за счет уменьшения размера профита. Однако убыточные сделки должны регулироваться очень внимательно.

\subsection*{Правило просадки}

Если общий риск уменьшается на 7\%, то размер задействованных средств автоматически уменьшается на 10\%, чтобы смягчить влияние рынка. Т.к. риск на сделку -- это процент от общего риска, то уменьшение сайза на сделку уменьшает потери в период убытков. Рассмотрим пример:

\bigskip

\setlength{\parindent}{0.5cm}
Общий риск = \$10 000\par
Риск на сделку = \$10 000 * 0.01 = \$100\par
Порог уменьшения на 10\% = \$10 000 -- 0.07 * \$10 000 = \$9 300\par

\bigskip
\setlength{\parindent}{0cm}
Предположим, что система дает серию из 7-ми 1\%-ых убыточных сделок. В результате общий риск упал до \$9 300, а размер задействованных средств уменьшается:

\bigskip

\setlength{\parindent}{0.5cm}
Новый Общий риск = \$9 300 -- 0.1 * \$9 300 = \$8 370\par
Новый Риск на сделку = \$8 370 * 0.01 = \$83.7\par
Пересмотренный Порог уменьшения на 10\% = \$8 370 -- 0.07 * \$8 370 = \$7 784.1\par

\bigskip

Как результат, получем уменьшение риска на сделку со \$100 до около \$80.

\bigskip

Иметь общее представление о своих ожиданиях [по сделке] -- это первый шаг на пути принятия верных решений [в торговле]. После того, как вы разберетесь в ваших ожиданиях, имеет смысл применить эти принципы в своей торговле. \textbf{Theta Trend} отслеживает все позиции в таблице и пересматривает ожидания после каждой сделки. Смысл отслеживания своих ожиданий в том, что это помогает определить размер потерь в сделке. Гораздо легче контролировать убытки по сделке, чем профит.

\bigskip

Примечание: Есть два важных урока, которые необходимо выучить из обсужденной выше темы риск-менеджмента. Во-первых, никогда не ставить слишком много на одну сделку. Вы можете терять 1\% от счета довольно много раз, прежде чем нагрянут проблемы. Во-вторых, правила риск-менеджмента \textbf{Theta Trend} определены таким образом, чтобы защитить ваш торговый капитал (risk equity). Если вы терпите убытки, более имет смысл уменьшить торговый сайз, чем увеличивать его. Сокращая свой торговый сайз, вы значительно уменьшаете вероятность серьезной просадки на счете.

\section*{Глава 4. Описание стратегии}
\label{chapter4}
\addcontentsline{toc}{section}{Глава 4. Описание стратегии}

Примечание: Особенности правил открытия/закрытия позиции описаны в~\hyperref[chapter5]{\ul{Главе 5}} и~\hyperref[chapter6]{\ul{Главе 6}}. Эта часть предназначена лишь дать общее представление о системе \textbf{Theta Trend}. Если вы увязли в правилах входа/выхода и их деталях, то перечитывание этой главы вернет вам ясность понимания сути системы.

\bigskip

Система продает вертикальные сперды вне денег (OTM) в направлении ожидаемого тренда. Выбираются опционы с дельтой от 15 до 40, и с количеством дней до экспирации от 30 до 90. Цель -- собрать большую часть премии проданного [кредитного] спреда: 80\% от полученного кредита. Ширина вертикальных спредов на ETF -- два страйка, а размер кредита не должен быть меньше \$0.40. Условие для risk/reward -- не хуже, чем 4 к 1.

\bigskip

Система предназначена для ETF с ликвидными опционами. Не смотря на то, что система также подходит для опционов на фьючерсы и индексы, малый сайз опционов на ETF удобен для риск-менеджмента. В торговле используются опционы нескольких некореллированных рынков, что способствует диверсификации и уменьшению волатильности портфеля. Примеры портфелей смотри в~\hyperref[chapter7]{\ul{Главе 7}}. Важно отметить, во время присутствия крайних настроений (эйфории, или паники), рынки  имеют тенденцию двигаться строго в одном направлении, и тогда имеет смысл уменьшить сайз своих позиций.

\bigskip

\textbf{Theta Trend} торгует опционами, в противоположность покупкам ETF. Смысл в использовании опционов в том, что если цена значительно не изменится, или будет стоять, мы можем закрыть позиции с малым убытком (или даже с небольшой прибылью). С течением времени вертикальный спред теряет в стоимости, даже если цена базового актива не меняется, и может быть выкуплен дешевле изначального кредита.

\bigskip

Для того, чтобы торговать в стиле следования за трендом, \textbf{Theta Trend} использует простой индикатор определения тренда. Индикатор используется для удаления субъективного мнения трейдера по поводу того, что представляет собой восходящий тренд и нисходящий тренд. \textbf{Theta Trend} не пытается предсказать будущий тренд, а принимает в расчет только текущее состояние для принятия решения торговать в лонг или в шорт.

\bigskip

Примечание: Изначально, система \textbf{Theta Trend} была разработана как ответ на разочарования, полученные от торговли \textit{Iron Condors}. \textit{Iron Condors} наиболее выгодны, когда цена БА остается замкнутом диапазоне в течение сделки, но часто требуют ``регулирование'', если цена движется против \textit{Iron Condors} (за границу диапазона). В \textbf{Theta Trend} применяется только одна регулировка, а именно -- закрыть позицию, когда тренд [предположительно] меняется.

\bigskip

Индикатор \textit{Average  True  Range  trailing  stop  reversal} определяет предполагаемый тренд. \textit{ATR TS} расчитывается с параметрами (5; 3.5). Пересечение над/под  линией 3.5 * \textit{ATR TS} сигнализирует о входе.

Изображние 4.1: На графике показан ATR TS на IWM. Заметьте, стоп перемещается вверх и вниз вместе с рынком и четко показывает точки разворота.

\bigskip

Система \textbf{Theta Trend} продает вертикальные спреды, где короткий опцион -- около ATR TS, до тех пор, пока приемлемы параметры риска и тренд продолжается. Когда же цена показывает разворот на основании ATR TS, все открытые позиции закрываются и начинается анализ новых позиций в противоположном направлении. Таким образом, \textbf{Theta Trend} почти всегда в рынке.

\bigskip

Убыточные позиции закрываются, если цена закрытия БА оказалась выше/ниже TS, или ниже SMA(50), когда расстояние до SMA больше, чем до точки безубыточности на дату экспирации. Выход из убыточных позиций также происходит при достижении или привышении максимального размера убытка (в долларах) в сделке.

\section*{Глава 5. Правила открытия позиции}
\label{chapter5}
\addcontentsline{toc}{section}{Глава 5. Правила открытия позиции}



















\section*{Глава 6. Правила закрытия позиции}
\label{chapter6}
\addcontentsline{toc}{section}{Глава 6. Правила закрытия позиции}

\section*{Глава 7. Примеры портфелей}
\label{chapter7}
\addcontentsline{toc}{section}{Глава 7. Примеры портфелей}

\section*{Глава 8. Адаптация системы}
\label{chapter8}
\addcontentsline{toc}{section}{Глава 8. Адаптация системы}

\section*{Глава 9. Профитные привычки}
\label{chapter9}
\addcontentsline{toc}{section}{Глава 9. Профитные привычки}


\appendix
\section*{Приложение: Чеклист на вход}
\label{appendix}
\addcontentsline{toc}{section}{Приложение: Чеклист на вход}












\end{document}
