\documentclass{scrartcl}
\usepackage[utf8]{inputenc}
\usepackage[english, russian]{babel}
\usepackage[unicode=true]{hyperref}
\usepackage[usenames, dvipsnames]{color}

\title{Theta Trend Trading System}
\author{userkilouser}
\date{September 2014}

\begin{document}

\maketitle
Theta Trend -- Частная торговая система


\tableofcontents

\section{Введение и допущения}
\label{intro}
\textcolor{White}{.}

Добро пожаловать и спасибо за покупку \textbf{Theta Trend}. \textbf{Theta Trend} -- это опционная трендовая механическая торговая система (МТС) с положительной тетой (мы продаем опционы). Это значит, что \textbf{Theta Trend} следует за трендом цены [базового актива]. Однако, вместо того чтобы покупать базовый актив, \textbf{Theta Trend} продает опционы вне денег (OTM). Продажа OTM опционов позволяет увеличить количество способов заработать деньги с этой системой. Мы в выигрыше при сильном тренде, слабом тренде, и даже, в некоторых случаях, когда тренд не формируется или ломается. Кроме того, система приносит прибыль с течением времени из-за распада проданных опционов.

\bigskip

Использование \textbf{Theta Trend} предполагает владение базовыми понятиями об опционах и ETF, однако совсем нет необходимости быть эскпертом по волатильности или техническому анализу. Если вы понимаете, как работают опционы и знаете, как построить вертикальный спред, - вы разберетесь в \textbf{Theta Trend}. \textbf{Theta Trend} предназначена быть практичной, легко реализуемой системой, которая генерирует прибыльные сделки с более высокой вероятностью, чем типичные трендовые системы.

\bigskip

\textbf{Theta Trend} не требует от вас следить за рынком весь день и, определенно, не является \textit{day trading} системой. Большинство сделок длиться от пары недель до месяца и за ними можно наблюдать и управлять в течении нескольких минут в день. Однако важность управления рисками и дисциплину в трейдинге нельзя недооценивать.

\bigskip

Программное обеспечение, которым я лично пользуюсь и очень рекомендую -- это \textbf{Thinkorswim} от \textbf{TD Ameritrade}. Качество их софта -- превосходное и все изображения в этом документе взяты из этой программы.

\bigskip

Успешный трейдинг - это скорее путь, чем цель. Надеюсь, что эта система сохранит вам несколько лет, не тратя их на разработку и разочаровывающие исследования. Это уже сделано мной. Если что-нибудь в системе кажется неясным, -- не стесняйтесь написать на \textcolor{Blue}{\href{mailto:info@thetatrend.com}{info@thetatrend.com}}, \textit{enjoy}.

\bigskip

-Dan\par
\textcolor{Blue}{\href{mailto:info@thetatrend.com}{info@thetatrend.com}}

\bigskip

\section{Вертикальные сперды}
\label{chapter1}

\bigskip

Вертикальный спред является строительным блоком системы \textbf{Theta Trend}. Вертикальный спред называется так из-за природы создания, если глядеть на опционную доску, и может быть дебитовым (вы покупаете спред), или кредитовым (вы продаете спред). Все сделки \textbf{Theta Trend} начинаются с OTM кредитных спредов, так что мы всегда продаем премию. Мы продаем OTM опционы чтобы избежать проблем, связанных с досрочным исполнением [опционов]. Хотя вероятность досрочного исполнения конкретно OTM опционов является довольно малой.

\bigskip

Изображение 1.1 демонстрирует опционную доску для IWM -- ETF для Russel 2000. На доске представлены: \textit{open interest} (открытый интерес), \textit{delta} (дельта), \textit{implied  volatility} (волатильность), \textit{extrinsic} и \textit{theta} (тета). Вертикальный спред строится на опционах с одной датой экспирации и включает в себя продажу близкого к деньгам опциона (ATM) и покупку другого опциона для хеджа риска далеко вне денег. На изображении подсвечены опционы в деньгах (ITM).

\bigskip

Изображение 1.1 Опционы с разными датами экспирации и их цены для IWM (ETF для Russel 2000).

\bigskip

\textbf{Theta Trend} использует медвежий колл-спред для шорта рынка, и бычий пут-спред для лонга. Суть спреда в том, что, если предполагается рост рынка, то мы продаем путы ниже рынка. Если предполагается падение рынка, то мы продаем коллы выше текущей цены.

\bigskip

Изображение 1.2 показывает пример вертикального спреда который может использовать \textbf{Theta Trend} чтобы встать в лонг по рынку. На изображении: красная линия представляет ценовой профиль (risk profile) позиции на день экспирации, а белая линия - на текущий день. В то время как линия рисков [и доходов] отражает позицию на момент экспирации, управление риском позиции на текущий день является нашей главной заботой. Смысл именно в использовании для управления позицией именно рисков текущего дня. Если вы управляете риском позиции текущего дня, вместо упования на будущее, то это поможет избежать серьезных убытков.

\bigskip

Изображение 1.2 Вертикальный бычий пут-спред в SPY (S\&P 500 ETF) демонстрирует эмуляцию длинной позиции [в базовом активе].

\bigskip

Позиция на Изображении 1.3 показывает пример короткого OTM вертикального колл-спреда, который есть бычья версия трейда, представленного на Изображении 1.2. Этот пример - реальная сделка. Вы можете видеть, что текущая рыночная цена TLT (ETF на 20+ летние казначейские облигации) -- около \$118 и позиция в одном контракте имеет прибыль примерно \$18. Один из основных принципов системы \textbf{Theta Trend} -- это торговать множество мелких позиций на некореллированных рынках.

\bigskip

Изображение 1.3 Реальный вертикальный бычий колл-спред в TLT.

\bigskip

Одна из первых особенностей, которую стоит упомянуть о вертикальном спреде -- это его низкий коэффициент риск/прибыль. На изображении, показанном выше, коэффициент риск/прибыль равен 1.6 / 0.4 или 4 к 1. Однако, компромисс в том, что вероятность получить прибыль довольно велика. Мы можем примерно оценить вероятность, что опционы окажутся в деньгах на момент экспирации, глядя на дельту в день продажи [спреда].

\bigskip

\textbf{Theta Trend}, как правило, продает опционы с дельтой 15-40. Опцион с дельтой 15 имеет 15\% шанс оказаться в деньгах на момент экспирации, что означает 100\% - 15\% = 85\% шанс его обесценивания к дате истечения [опциона]. Все в работе с опционами связано компромиссом. Компромисс при продаже опционов далеко вне денег в том, что вы получаете небольшие премии, но с большой вероятностью  обесценивания [опционов к экспирации]. Малые величины [получаемой] премии могут быть из-за неблагоприятного соотношения риск/прибыль или, в некоторых случаях, когда открытая позиция не соответствует критерию риск/прибыль системы \textbf{Theta Trend}. Специфические характеристики отношения риск/прибыль, которые используются системой \textbf{Theta Trend} для открытия позиции обсуждаются в Главе 6 - Правила открытия позиции.

\bigskip

\section{Принципы следования за трендом}
\label{chapter2}

\bigskip

\textbf{Theta Trend} -- это трендовая торговая система. Трендовики обычно зарабатывают, держа прибыльные позиции до окончания тренда, и быстро выходят из убыточных позиций. Большинство трендовиков придерживаются нескольких типов механических торговых систем со строгим управлением рисками и правилами открытия и закрытия позиций. Как говорится, большинство трейдеров пытается следовать за каким-либо типом  тренда. Тренд может образоваться на любом тайм-фрейме -- от секунд до месяцев или лет. Успех большинства трендовых систем базируется на том понятии, что цена актива не адекватна (not normally 
distributed), а должна быть много выше или ниже.

\bigskip

Многие трендовые МТС имеют меньше прибыльных сделок, чем убыточных, но все-таки приносят деньги, т.к. доход от прибыльных сделок много больше убытков. Отношение прибыльных сделок к убыточным как 40\% к 60\% -- это не редкость, и может быть встроено в профитную торговую систему с надлежащим управлением рисками. Система \textbf{Theta Trend} построена так, что добивается увеличенного процента прибыльных сделок. Таким образом, использовать систему психологически легче. \textit{Forward testing} системы \textbf{Theta Trend} показал, что соотношение прибыльных сделок к убыточным как 60\% к 40\% вполне достижимо.

\bigskip

Сделки в системе \textbf{Theta Trend} имеют ограниченный потенциал по прибыли, но и убытки уменьшены и даже возможен доход, если ситсема стоит против рынка. \textbf{Theta Trend} призвана быть механической системой, т.е. мнение трейдера не включено в процесс принятия решения [по сделке]. Сделка либо соответствует критериям открытия/закрытия, либо нет. Тренд выявляется индикатором системы, и сделки не открываются против тренда. \textbf{Theta Trend} также предназначена быть в рынке все время, пока вход валиден и нет причин для закрытия.

\bigskip

Не смотря на то, что система \textbf{Theta Trend} не включает в себя мнение трейдера, основанного на [внутри?]дневных данных, все-таки необходимо принятие решение по поводу портфеля инструментов для торговли. Трендовые системы работают на многочисленных некоррелированных рынках т.к. это увеличивает и выравнивает доходы. \textbf{Theta Trend} поддерживает идею, что торговля на некоррелированных рынках увеличивает доходы, и работает на широком спектре рынков: валютном, рынке облигаций, товаров и акций. Примеры портфелей обсуждаются в последующих главах. Главная идея -- это торговать на нескольких некоррелированных друг к другу рынках в надежде, что если вы ошиблись и теряете деньги на одном рынке, а на другом рынке (или ранках) вы правы и это перекрывает убыток.














\end{document}
